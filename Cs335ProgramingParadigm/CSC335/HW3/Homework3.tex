\documentclass{article}
\usepackage[utf8]{inputenc}
\usepackage[margin=1in]{geometry}
\usepackage{listings,setspace,mathtools,courier}

\doublespacing{}

\setlength{\parindent}{0pt}

\begin{document}

Nelson Batista

CSC 33500 Programming Language Paradigms --- Section R

9/22/16

Homework 3

\section*{1.29}

A solution is given below:

\lstset{basicstyle=\ttfamily}
\lstinputlisting[language=Lisp]{ex29.scm}

Running \texttt{simpson} with \texttt{cube}, \texttt{0}, and \texttt{1} as the arguments for \texttt{f}, \texttt{a}, and \texttt{b}, respectively, returns 0.25 for both \texttt{n = 100} and \texttt{n = 1000}, so we know the function definitely approximates the correct value for the definite integral of the function $f(n) = n^3$ between 0 and 1.

\section*{1.30}

The substitutions have been made in the following code:

\lstinputlisting[language=Lisp]{ex30.scm}

\section*{1.31}

\subsection*{a}

The \texttt{product} function can be implemented like so:

\lstinputlisting[language=Lisp, linerange=1-5]{ex31.scm}

We can use this function to calculate factorials almost trivially, since a factorial of $n$ is just the multiplication of all the numbers from 1 to $n$. Here is the code:

\lstinputlisting[language=Lisp, linerange=7-10]{ex31.scm}

\end{document}
