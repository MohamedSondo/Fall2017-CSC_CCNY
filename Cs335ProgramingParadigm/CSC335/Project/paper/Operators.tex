\documentclass{article}
\usepackage[utf8]{inputenc}
\usepackage[margin=1in]{geometry}
% sharelatex doesn't seem to render the color here but whatever
\usepackage[urlcolor=cyan]{hyperref}
\usepackage{amssymb,listings,setspace,mathtools}

\doublespacing{}
\DeclareMathOperator\triplebar{|||} % for interleaving operator
\begin{document}

\begin{center}
\section*{Communicating Sequential Processes in R5RS Scheme}
\subsection*{CSC 33500 Programming Language Paradigms --- Fall 2016}
\subsection*{Professor Douglas Troeger}
\subsection*{by Nelson Batista and Irving Derin}
\end{center}

\subsection*{Introduction}

Concurrency is a very important method for structuring computer programs to
allow them to perform various actions as if they were instead acting as a set
of individual processes running in parallel. Running programs in parallel
allows them to execute overall more quickly, since neither is waiting for
the completion of the other. However, whereas parallelism involves running
multiple programs (or processes) at precisely the same moment, concurrency
is a method of structuring programs into more individual units which may run
in parallel or may run at differing times. While this difference sounds
trivial, concurrency allows for greatly increased efficiency regardless of
whether the individual components are actually running in parallel.

For instance, \textit{Go} has been described as ``a concurrent [programming]
language'' by one of its creators, Rob Pike.
\href{https://www.youtube.com/watch?v=cN_DpYBzKso}{His example} 
is that of a gopher attempting to burn a large pile of outdated books by 
putting them into a wheelbarrow and delivering them to an incinerator, then
going back to get more books and repeat the process until the pile of books
is empty.

In this case, concurrency would be the introduction of one or more gophers
to the process of the book-burning, each acting independently of the
others. One gopher may be the wheelbarrow gopher, moving the wheelbarrow
full of books to the incinerator. Another gopher may be the incinerator
gopher, emptying the wheelbarrow into the incinerator. Still another
gopher may be the book gopher, responsible for loading books from the 
pile to the wheelbarrow. Pike notes that the design of each gopher must be
careful not to have it prevent the others from doing its job. The execution
of the book-burning program, so to speak, is represented by each gopher's
movement as it performs its respective task. While each gopher may be 
running with the others at the same time (in parallel), it is not necessary
in order for the concurrent nature of the program structure to be
maintained. Additionally, assuming each gopher has been implemented 
properly, the overall book-burning program will run much more efficiently,
regardless of whether the gophers are actually running in parallel (which
may be impossible to truly be the case if, say, the program is running
on a single-core CPU).

This idea of ensuring that each gopher is properly coordinated with the
others is itself an important aspect of concurrency. There are several
methods of implementing proper coordination. Among them is the idea of
communicating sequential processes (CSP). CSP is a language defined by 
Anthony Hoare in his 1977 paper \textit{Communicating Sequential Processes}.

In it, he describes a language in which objects and various other parts of
the program are modeled as processes which can set off and react to events.
These events are communicated to other processes through channels. This
idea, further explored in the paper 2011 \textit{Places: Adding Message-Passing Parallelism to Racket} by Kevin Tew et al., which
defines its own set of functions for creating and acting on different
processes which can pass messages among themselves and run in parallel.
This along with the lack of similar work in Scheme inspires the possibility
of implementing Hoare's CSP in Scheme in a similar manner.

We first begin with the most fundamental building blocks: processes and 
channels.

\subsection*{The Process Primitive}

The process primitive is a bit poorly-defined, as it is not itself a
function, but a way of modeling other functions as processes which can
run concurrently.

\subsubsection*{Implementation of the Queue Data Structure}

The most important aspect of our implementation of the process primitive
was the implementation of a process queue. This necessarily requires 
implementation of a queue data structure, which is not normally available
in Scheme (lists are essentially stacks, as push is represented by cons
and pop by car).

While the queue as a concept should be familiar to most of our audience,
we describe it in passing as a data structure onto which multiple values
may be placed, and values in it are accessed in the order which they were
placed onto the queue. The first element placed into the queue by a push
operation is the first element retrieved by a pop operation. The queue
is thus an example of a ``first in, first out'' data structure. R5RS's
use of what are essentially stacks for its sole data structure makes 
an implementation of queues a bit non-obvious, as opposed to other 
languages which offer data structures with random access.

Our code, adapted from the implementation \href{http://yazary.blogspot.com/2012/02/implementing-queue-in-scheme_22.html}{here}, contains all the various functions expected from a queue, with
the exception of \texttt{full?}.

Queues are declared by the \texttt{make-queue} function. This initializes
a queue with an empty list as its list of values, and an empty list as the
most recent value put in the queue (to implement the \texttt{rear} 
function).

The \texttt{push} function takes a value and a queue, and appends the
value to the queue. The way it does this is by using the \texttt{set-cdr!}
function provided in R5RS to change the cdr of the queue to the value
contained within a list. This creates a new queue with the pushed value
as the last element in its list of values. The \texttt{push} function also
sets the cdr of the queue to the value pushed so that it can be accessed
in constant time by the \texttt{rear} function.

\subsection*{The Prefix Operator}

The prefix operator takes an event and a process. The operator creates a process which performs the event given, then behaves
like the process given. For instance, $(x \rightarrow P)$ (``x then P'') refers to the process which first completes event x then behaves like process P.

In his text, Hoare offers several examples to better illustrate potential uses of the operator as well as to clarify its functionality. Perhaps the simplest of 
these is his proposed clock process:

\[ CLOCK = (tick \rightarrow CLOCK) \]

This defines a clock process recursively. It is a process which simply ticks, then behaves like a clock process, which also simply ticks, then behaves like a clock
process, and so on in an infinite loop. The result is a process which ticks indefinitely without doing anything else. Below is a similar clock process, a clock 
which ticks, then tocks, then behaves like a clock process:

\[ CLOCK = (tick \rightarrow (tock \rightarrow CLOCK)) \]

This demontstrates the possibility of nested prefix operators. The clock process is now defined as a process which ticks, then behaves like a process which 
``tocks'', so to speak, then behaves like a clock. We thus have a process which alternates ticking and tocking, more inline with the idea of a physical clock.

In our implementation, it was quite trivial to think of an event as simply the execution of a single function. As such, the implementation of the prefix operator
follows naturally as a function which executes some \texttt{event} argument before adding a separate \texttt{proc} argument to the process queue to be executed
like any other process. The function is thus defined as \texttt{(prefix event proc)} and, exactly as described, takes an event and a process as arguments and 
performs the event (executes the function passed as \texttt{event}) before adding the process to the event queue to be executed. 

Thus, a process can be define using the prefix operator by running \texttt{(coroutine (prefix x p))}, which pushes to the process queue a process which performs
some action \texttt{x}, then behaves like another proccess \texttt{p}, exactly as described.

There are a number of limitations to this approach, however. If the event passed happens to write to a channel, for instance, it is necessary for at least one 
other process be in the process queue. This is because the function \texttt{write-channel!} calls \texttt{start}, which pops the process queue and applies the
element popped. If the event writes to a channel (perhaps defined outside so that it is contained within the event's closure, or defined within the event function
itself) without any processes in the queue, it will call \texttt{start}, and an error will be thrown as an empty list is not a function which can be applied. 

One possible way to circumvent this problem is by pushing the process to the queue before doing the event. This would guarantee that at least one process is on the
queue before the event is run, preventing any single \texttt{write-channel!} call within the event from causing problems. However, this approach would not only 
violate the premise of the prefix operator in that the event is to be performed \textit{before} behaving like the given process, but would also do nothing to guard
against more than a single \texttt{write-channel!} call within the event. It is thus more practical, if more dangerous, to leave the order as-is. This keeps the 
implementation faithful to Hoare's description of the prefix operator, but requires a precondition that the event passed will either not make calls to 
\texttt{write-channel!} or will make sure that the process queue has at least one process in it before making any such calls.

\subsection*{The Deterministic Choice Operator}

The deterministic choice operator is a bit more complex. It is technically a combination of two processes defined by prefix operators. The exact notation is:

\[ (x \rightarrow P)\  \square \  (y \rightarrow Q) \]

This is a process which can perform one of two possible actions. Depending on which action is completed first, that action's corresponding process will become the behavior of the process.

We can consider a simplified version of an example provided by Hoare. We can model a vending machine as a process which waits for a customer to select
a product to dispense. Say the vending machine offers chips and candy. In this case, the customer selecting chips represents event $x$, while $P$ is represented by
a machine which dispenses a single bag of chips before behaving like a vending machine (in that it waits for an input before dispensing a single item). The 
situation is similar for candy. We note that this demonstrates the ability of deterministic choice to be defined recursively in a similar manner to prefixing.
$P$ and $Q$ are both processes which themselves contain determinstic choice. Of course, this is partially due to them being essentially the same process, but the
idea is that it is a possibility.

Implementing deterministic choice in Scheme proved to be more challenging than would immediately seem. Ultimately, the design settled on was that of a function 
which takes \texttt{7} arguments: the two events, their respective processes, two symbols, and a channel. The choice is therefore ultimately left to the environment.
The channel is used to read a symbol. For this, we use \texttt{peak-channel} rather than \texttt{read-channel!} so as to not modify the channel in the process of
checking whether its value is equal to the symbol passed to the operator. If it is, then the process defined by the prefix operator applied to the symbol's corresponding event and process is pushed to the process-queue. Otherwise, the operator pushes an instance of itself being called with the same arguments to the process queue. This, in a sense, allows the process to wait for one of the two symbols to be sent on the channel. Thus, we model the event deciding which process is to determine the behavior of the deterministic choice operator as a combination of a particular value being read from a channel and an event being executed through the prefix operator's normal functionality. This leaves the actual choice up to the environment (specifically, the execution of other processes) in which the operator is being used, since they are responsible for sending one of the two necessary symbols on the channel, and thus responsible for allowing the deterministic choice operator to do anything other than continuously push itself onto the process queue. 

It may appear that the order in which the channel's first entry is checked would favor the first process's execution, since the first process is the one that has its value checked for presence on the channel. However, our implementation of channels ensures that
only one value is ever on the channel at once, thus preventing the case in which both process' symbols are on the channel at once, which would stop one process from
ever being able to enter. On the other hand, this introduces a separate issue in that, since the operator's method of waiting involves placing itself back on the
process queue, there is no guarantee that a process won't write a value to the channel, expecting the write process to set off the deterministic choice, only to have
the subsequent \texttt{start} call execute a different process.

\subsection*{The Non-Deterministic Choice Operator}
\subsubsection*{Random Number Generator}
Due to the non-deterministic nature of this operator, it proved useful to implement a (very) crude psuedo-random number generator. This was tricky, however, since R5RS 
does not provide any functions to generate random numbers, nor does it provide access to system variables, such as the current time, which could be used as an
acceptable seed to an algorithm which generates them. As such, it was a bit of a challenge to find a value which could be used as a seed while also being volatile
enough to not generate numbers which are predictable in their randomness. For instance, values such as the length of the host operating system name (if it were available), would be a poor choice for such a generator, as any particular system would always generate the same numbers in the same order since the seed would only
change if the operating system itself were changed. Ultimately, we settled on using the length of the process queue, which changes quite frequently, especially in
any real-world application of our CSP implementation. While this is \textit{far} from optimal, for our purposes and the limits of the language it had to suffice.

As for the algorithm itself, in the interest of simplicity, we chose to implement a linear congruential generator. This generator is very simple. Starting from
a seed value $X_0$, the linear congruential generator generates numbers using the recursive formula $X_{n+1} = (aX_n + c) \mod m$, where $a$, $c$, and $m$ are constants.
In the interest of not digressing too far from the project itself, we will omit a more detailed explanation and instead explain that, for this particular generator,
we used the same constants used by glibc, with $a = 1103515245$, $c = 12345$, and $m = 2^{31}$.

\subsubsection*{Non-Deterministic Choice}
The non-deterministic choice operator is similar to its deterministic counterpart, in that the environment provides a particular event and,
in doing so, sets off a particular process, but the key difference
is that the environment ultimately cannot predict which process will begin. The notation is:

\[ (a \rightarrow P) \sqcap (b \rightarrow Q) \]

Thus it is also a combination of prefix operators. The operator only responds if both a and b are received, at which
point one of the two processes will be selected. For this, we used
the random number generator. If the generator returns a 0, the first
process is selected and placed on the process queue. Otherwise, the
generator returned a 1, and the other process is selected instead.

To achieve this, within the function \texttt{non-determ-choice}, we
defined a helper function called \texttt{phase2}. When
\texttt{non-determ-choice} checks the channel and sees that its
current value is equal to one of the two symbols passed, the other
symbol, along with the remainder of \texttt{non-determ-choice}'s 
arguments to \texttt{phase2} and places it in the process queue.
When \texttt{phase2} runs, it checks the channel for the other value
and, if it is on the channel, uses the random number generator
to generate a 0 or a 1 (by calling \texttt{(modulo (rng) 2)}), and
from that selects one of the two event-process pairs to turn into
a prefix-type process before placing it in the process queue.

If, at any time, the process reads the channel and does not find one
of the necessary values, the process is restarted by placing 
\texttt{non-determ-choice} back in the process queue with its 
initial arugments. This holds even if \texttt{phase2} is the process
checking the channel, since the non-deterministic-choice need not
do anything if \textit{both} values are not sent on the channel, so 
we may simply wait for them.

\subsection*{The Interleaving Operator}
The interleaving operator proved to be a more significant challenge than
it may seem. Its description is quite simple: it takes two processes and
forms a process which executes actions from both processes concurrently.
The order of their interweaving is undefined. However, this presents a very
interesting challenge. It requires our system to, in some way, have some
knowledge about the processes themselves.

The operator is denoted by:

\[ P \triplebar Q \]

As described, this creates a new process which executes actions of both
processes in arbitrary order. This would thus suggest that we need to
implement not only some type of meta-concurrency (while we have implemented
concurrency, there is still usuaully one process running in its entirety
at a time, though the processes should be designed around both this and the
provided operators and support for channels). However, we note that it is
not necessary for a given process to finish executing entirely before 
another process begins execution. Indeed, each call to \texttt{write-channel!} starts execution of a new process. Thus, through
clever use of \texttt{write-channel!} calls and perhaps some shenanigans
involving continuations, it may be possible to perhaps place the two 
processes on the process queue, before calling \texttt{write-channel!} 
with some dummy value to immediately begin execution of whatever is in the
queue. While there are no guarantees that the process executing will be
the one pushed to the queue (if there is so much as a single other process
on the queue at the time of pushing, that will be run instead), so we were
ultimately unable to attain a satisfactory implementation of this 
particular operator. This is left as an exercise to the reader.

\subsection*{Hiding}

Hiding is another operator that proved to be a bit out of our scope. The
hiding operator is, in a sense, the opposite of the prefix operator. 
Whereas the prefix operator creates a process which executes a particular
event before behaving like another process, the hiding operator prevents
certain events from occurring. Hoare provides an excellent example in the
``noisy'' vending machine. The machine normally makes several sounds,
``clink'' and ``clunk'' during its operation, in addition to the normal
behavior of a vending machine. For example, the noisy vending machine, 
which accepts a coin before dispensing chocolate, may be modeled as:

\[ NOISYVM = (coin \rightarrow (clink \rightarrow (clunk \rightarrow (chocolate \rightarrow NOISYVM)))) \]

However, should we want a vending machine which simply accepts a coin 
before dispensing chocolate (without making any noise), we may describe one
using the hiding operator like so:

\[ VM = NOISYVM\  \backslash\  \{clink,clunk\} \]

This is equivalent to:

\[ VM = (coin \rightarrow (chocolate \rightarrow (NOISYVM\ \backslash \  \{clink,clunk\}))) \]

Or, substituting our previous definition,

\[ VM = (coin \rightarrow (chocolate \rightarrow VM)) \]

As expected. The difficulty in implementing such an operator should now
be apparent. Hoare offers an implementation which relies on knowledge of 
the process's alphabet, which our implementation of processes does not 
provide. Specifically, one may implement a function which takes both the
process and the set of its alphabet's events which must not occur as 
arguments, then check each action the process takes to see if it is among
the set of events which must not occur, and force the event to either
not occur or occur in such a way that its effects are invisible to the 
system. As our implementation of processes does not concern itself with
alphabets, and R5RS provides little in the way of reflection, such an
implementation cannot be made without substantial effort towards rebuilding
our CSP implementation.
\end{document}
