\documentclass{article}
\usepackage[utf8]{inputenc}
\usepackage[margin=1in]{geometry}
\usepackage{listings,setspace,amsmath,courier}

\doublespacing{}

\setlength{\parindent}{0pt}

\begin{document}

Nelson Batista

CSC 33500 Programming Language Paradigms --- Section R

9/20/16

Homework 2

\section*{Abelson and Sussman}

\subsection*{1.11}

\vspace{\baselineskip}

$ f(n) = \begin{cases}
  n & n < 3 \\
  f(n-1) + 2f(n-2) + 3f(n-3) & n \geq 3
\end{cases}
$

\vspace{\baselineskip}

A recursive solution is given below:

\lstset{basicstyle=\ttfamily}
\lstinputlisting[language=Lisp, linerange=1-5]{ex11.scm}

Proof:

We first see that, for any value $n < 3$, \texttt{function-recur n} returns $n$, due to the first \texttt{cond} expression. For our induction hypothesis, we assume that, for some arbitrary value $k$, \texttt{function-recur k} correctly returns $f(k-1) + 2f(k-2) + 3f(k-3)$. The recursive calls in the \texttt{cond}'s \texttt{else} expression exhibit exactly this behavior. Our goal is thus to show that \texttt{function-recur} will return the correct value for $k+1$, or that $f(k+1) = f(k) + 2f(k-1) + 3f(k-2)$. We have:

\[ f(k+1) = f((k+1)-1) + 2f((k+1)-2) + 3f((k+1)-3) \]
\[ f(k+1) = f(k) + 2f(k-1) + 3f(k-2) \]

The last statement shows that, assuming \texttt{function-recur k} returns the correct value, so will \texttt{function-recur (+ k 1)}.

An iterative solution is given below:

\lstinputlisting[language=Lisp, linerange=45-52]{ex11.scm}

Proof: 

We see that the first line of the function will have it return $n$ in the event that $n <= 3$. That behavior is thus fulfilled. When $n \neq 3$, we have a call to a ``helper'' function that handles the iteration. This function, \texttt{function-tail} is tail-recursive. The \texttt{result} parameter holds the current result of the function call, which is accurate if we take $n$ to be the current value of \texttt{count} at any given point in the computation. This is guaranteed at all points of the iteration. When \texttt{function-tail} is first called, it is passed the arguments \texttt{n = n, count = 2, result = 2, a = 1, b = 0}. We see the initial value of \texttt{result} is indeed the correct value of $f(\texttt{count})$, 2. The function iterates, and at each step, the values of $f(\texttt{count}-2)$ and $f(\texttt{count}-3)$ are stored and updated in \texttt{a} and \texttt{b}, respectively. 
At \texttt{count = 3}, the value of \texttt{result} is \texttt{2 + 2*(1) + 3*(0)}, which is 4 and therefore the correct value for $f(3)$. If $n \neq 3$, then the function is called again with an incremented value of \texttt{count}, and \texttt{a} is replaced with the current value of \texttt{result} to remain consistent with our observation that \texttt{a} is the value of $f(n-1)$, and \texttt{b} is replaced with the current value of \texttt{a} to contain the value of $f(n-2)$.

This chain continues upward until $\texttt{count} = n$, at which point the function will return the current value of \texttt{result}, which we have demonstrated will contain the correct value of $f(n)$.

\subsection*{1.12}

A recursive solution to calculate an element of Pascal's triangle given a particular row and position within the row is given below:

\lstinputlisting[language=Lisp, linerange=12-15]{ex11.scm}

Proof:

Note that a particular element in a particular row of Pascal's triangle can be found using 

$f(row, elem) = \begin{cases}1 & row <= 2\ or\ elem = 1\ or\ elem = row \\
                            f(row-1, elem-1) + f(row-1, elem) & otherwise
                \end{cases}$

First we observe that the function returns the correct value for any element along the edges of the triangle. If the element asked for within a given row is at position 1, or at the last position in the row (a given row $n$ in Pascal's triangle has $n$ elements in it), the function returns 1, due to the first expression in the \texttt{cond} statement. This is correct behavior, as even the question itself notes that any element on the edge of a row will be 1.

To prove that the function works for all elements, we first assume it works for any arbitrary element $k$ in any arbitrary row $r$. We have already demonstrated that, no matter what value we set for $r$, the function is correct if $k$ happens to be the first or last element in the row. Beyond that, we must show that, assuming \texttt{pascal r k} correctly returns \texttt{(+ (pascal (- r 1) (- k 1)) (pascal (- r 1) k))}, it will also return the correct value for $r+1$ and $k+1$. That is, \texttt{pascal (+ r 1) (+ k 1)} should return the sum of \texttt{pascal r k} and \texttt{pascal r (+ k 1)}, the sum of the element before $k$ in the row above $r$ and the element at position $k+1$ in the row above $r+1$.

\begin{center}
  $\texttt{pascal (+ r 1) (+ k 1)} = \texttt{(+ (pascal (- (+ r 1) 1) (- (+ k 1) 1))}$
  
  \hspace{1.85in}$\texttt{(pascal (- (+ r 1) 1) (+ k 1)))}$

  $\texttt{pascal (+ r 1) (+ k 1)} = \texttt{(+ (pascal r k) (pascal r (+ k 1)))}$
\end{center}

Thus, \texttt{pascal (+ r 1) (+ k 1)} returns the correct value for some arbitrary element $k$ in some arbitrary row $r$.

\section*{Sum of digits}

A recursive solution to return the sum of digits in a non-negative integer is below:

\lstinputlisting[language=Lisp, linerange=17-21]{ex11.scm}

Proof:

First, we note that, if a number $n$ is less than 10, the sum of its digits is simply $n$, since it has only that digit. This is satisfied by the second expression in the \texttt{cond} statement, with the first expression covering the precondition that our input will not be a negative integer. We then assume that the function returns the correct sum of the digits of an arbitrary number $k$, equal to the sum of \texttt{(sum-of-digits-recur (quotient k 10))} and \texttt{(modulo k 10)}. The function will be called with progressively fewer digits until only a single one remains, at which point it will return that digit's value and travel up the stack, adding the rest of the digits through the deferred addition. We assume $\texttt{(sum-of-digits-recur k)} = \texttt{(+ (sum-of-digits-recur (quotient k 10)) (modulo k 10)}$. We must then prove 

$\texttt{(sum-of-digits-recur (+ k 1)} = \texttt{(+ (sum-of-digits-recur (quotient (+ k 1) 10))}$

$\texttt{(modulo (+ k 1) 10)}$. This is a trivial substitution.

The iterative solution:

\lstinputlisting[language=Lisp, linerange=23-29]{ex11.scm}

\section*{Increasing order of digits}

A recursive solution to return whether a given number's digits are in increasing order is given below:

\lstinputlisting[language=Lisp, linerange=31-34]{ex11.scm}

An iterative solution:

\lstinputlisting[language=Lisp, linerange=36-43]{ex11.scm}

\end{document}
